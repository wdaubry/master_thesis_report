\chapter{Conclusion}

This master's thesis focused on the single anchor single tag \gls{uwb} localization with the aim of studying the possible of such implementation. Based on simulations and an experimental composed of an \gls{uwb} transmitter, a micro-controller and an Android application.
\vspace{2mm}

This work was divided in three principal phases. First, an algorithm performing peak extraction from a \gls{cir} was developed. Based on those peaks, two different algorithms performing the localization based on a \gls{cir} originating from a single anchor were presented, one based on the trilateration problem and another focused on \gls{cir} comparison through the arrival time of the extract peaks.
\vspace{2mm}

The second part focuses on the simulation implemented to test the different algorithms developed in the previous chapter. This chapter present the advantages and the drawbacks of both algorithms as well as their limitations. A discussion has been made on the choice of the anchor location.
\vspace{2mm}

Test last part focuses on the hardware implementation of the two methods by enabling the \gls{cir} recuperation on the DWM1000, transferring to the smartphone through the PSoC and storing it in the cellphone memory. 
\vspace{2mm}

In conclusion, the results obtains in the chapter \ref{simulations} show some severe limitation to the one anchor localization algorithms implemented. Most of those limitation are due to the unknown change of the room and the symmetry occurring in the room. To assess the validity of the simulations presented in this thesis, an experimental validation should be mandatory. 
\vspace{2mm}

About the evolution perspectives, the next step would be to experimentally test the algorithms developed in this thesis to assess their quality. We could also enhance or modify the peak extraction algorithm, developed some new localization algorithm. What could also be interesting is to based a further implementation based on the history of the locations as in \cite{meissner2010uwb}.